%
%   This program is free software: you can redistribute it and/or modify
%   it under the terms of the GNU General Public License as published by
%   the Free Software Foundation, either version 3 of the License, or
%   (at your option) any later version.
%
%   This program is distributed in the hope that it will be useful,
%   but WITHOUT ANY WARRANTY; without even the implied warranty of
%   MERCHANTABILITY or FITNESS FOR A PARTICULAR PURPOSE.  See the
%   GNU General Public License for more details.
%
%   You should have received a copy of the GNU General Public License
%   along with this program.  If not, see <http://www.gnu.org/licenses/>.
%

% Version: $Revision: 8032 $

\begin{thebibliography}{999}
	% to make the bibliography appear in the TOC
	\addcontentsline{toc}{chapter}{Bibliography}

	\bibitem{jalali-Leake13-2}
	Jalali, V., Leake, D.:
	\newblock Extending case adaptation with automatically-generated ensembles of
	  adaptation rules.
	\newblock In: Case-Based Reasoning Research and Development, \uppercase{ICCBR}
	  2013, Berlin, Springer (2013)  188--202

	\bibitem{mantaras-et-al05}
	{Lopez de Mantaras}, R.; McSherry, D.; Bridge, D.; Leake, D.; Smyth,
	  B.; Craw, S.; Faltings, B.; Maher, M.; Cox, M.; Forbus, K.; Keane, M.;
	  Aamodt, A.; and Watson, I.
	\newblock 2005.
	\newblock Retrieval, reuse, revision, and retention in \uppercase{CBR}.
	\newblock {\em Knowledge Engineering Review} 20(3).

	\bibitem{aha91}
	Aha, D., Kibler, D.:
	\newblock Instance-based learning algorithms
	\newblock Machine Learning (1991) 37--66

	\bibitem{hanney-keane97}
	Hanney, K., Keane, M.:
	\newblock The adaptation knowledge bottleneck: How to ease it by learning from
	  cases.
	\newblock In: Proceedings of the Second International Conference on Case-Based
	  Reasoning, Berlin, Springer Verlag (1997)  359--370

\end{thebibliography}
